\documentclass[a4paper,12pt]{article}

\usepackage{packages/general}
\usepackage{packages/boxes}


\title{\textbf{Math Lecture Notes Template}}
\author{Angad Kapoor}
\date{}

\begin{document}
\maketitle
\begin{abstract}
    Separate colored box environments for each section (with proofs), made using the \verb|tcolorbox| package.
\end{abstract}
\tableofcontents

\linediv

\thispagestyle{empty}

\section{Definitions}
\begin{definition}{Cauchy Sequences}{ndist}
    A sequence $\{a_n\}$ of real numbers is a \textbf{Cauchy sequence} if $\forall \epsilon > 0$, $\exists N\in\N$ such that $|a_m - a_n| < \epsilon$ $\forall m,n \geq N$.
\end{definition}

\section{Theorems}
\begin{theorem}{Differentiability implies Continuity}{}
    If a function $f$ is differentiable at a point $a$, then it is continuous at $a$.
\end{theorem}
\begin{tproof*}{}{}
    We want to show that $\lim\limits_{x\to a}f(x) = f(a)$. 
    First, notice that $f(x)$ can be rewritten as $f(x) = f(a) + \dfrac{f(x) - f(a)}{x-a}(x-a)$. 
    Since $f$ is differentiable at $a$, we know that $\lim\limits_{x\to a}\dfrac{f(x) - f(a)}{x-a}$ exists and is finite.
    $x$ is continuous, so $\lim\limits_{x\to a} (x-a) = 0$, so $f(x) \xrightarrow{x\to a} f(a) + f'(a)\cdot 0 = f(a)$. $\blacksquare$
\end{tproof*}

\section{Propositions}
\begin{proposition}{}{}
    A Cauchy sequence with a convergent subsequence converges.
\end{proposition}
\begin{pproof*}{}{}
    Let $\{a_n\}$ be a Cauchy sequence, and let $\{a_{k_n}\}$ be a convergent subsequence of $\{a_n\}$.
    Let $\lim\limits_{n\to\infty}x_{a_{k_n}} = a$.
    Let $\epsilon > 0$.
    Since $a$ is the limit of the subsequence, we know $\exists N_1\in\N$ such that $|a - a_{k_n}| < \frac{\epsilon}{2}$ $\forall n\geq N_1$.
    Similarly, since $\{a_n\}$ is Cauchy, we know $\exists N_2\in\N$ such that $|a_m - a_n| < \frac{\epsilon}{2}$ $\forall m,n\geq N_2$.
    Choose $N = \max\{N_1, N_2\}$.
    Then, for $n\geq N$, we know $k_n \geq n \geq N$ (by the definition of a subsequence). 
    So, $|a - a_n| \leq |a - a_{k_n}| + |a_{k_n} - a_n| < \frac{\epsilon}{2} + \frac{\epsilon}{2} = \epsilon$.
    Then, by the definition of the limit, $\{a_n\}$ converges to $a$. $\blacksquare$
\end{pproof*}

\section{Corollaries}
\begin{corollary}{}{}
    A Cauchy sequence converges.
\end{corollary}
\begin{cproof*}{}{}
    Every Cauchy sequence has a convergent subsequence, and we just saw that every Cauchy sequence with a convergent subsequence converges.
    Hence, every Cauchy sequence converges.
\end{cproof*}

\section{Lemmas}
\begin{lemma}{}{}
    This is a lemma.
\end{lemma}
\begin{lproof*}{}{}
    This is the proof. 
\end{lproof*}

\section{Examples}
\begin{example}{Using Cesaro-Stolz}{}
    Show that the sequence defined by $x_n = n^\frac{1}{n}$ converges, and find its limit. \\

    We can define $x_n$ as $x_n = a_n^{\frac{1}{n}}$, where $a_n$ is another sequence $a_n = n$.
    Thus, $\frac{a_{n+1}}{a_n} = \frac{n+1}{n} = 1 + \frac{1}{n}$, which converges to $1$ as $n\to\infty$.
    Then, by the Cesaro-Stolz theorem, $\lim\limits_{n\to\infty}x_n = \lim\limits_{n\to\infty}a_n^{\frac{1}{n}} = \lim\limits_{n\to\infty} \frac{a_{n+1}}{a_n} = 1$.
\end{example}

\end{document}